\documentclass[11pt]{article}

\newcommand{\numpy}{{\tt numpy}}    % tt font for numpy

\topmargin -.5in
\textheight 9in
\oddsidemargin -.25in
\evensidemargin -.25in
\textwidth 7in

\usepackage{amsmath,amsthm,amsfonts,amssymb,amscd}
\usepackage{lastpage}
\usepackage{enumerate}
\usepackage{fancyhdr}
\usepackage{mathrsfs}
\usepackage{xcolor}
\usepackage{graphicx}
\usepackage{listings}
\usepackage{hyperref}
\usepackage{enumitem}
\usepackage{float}
\usepackage{fancyvrb}

\begin{document}

% ========== Edit your name here
\author{Palle Morris e1160346\\ Elmenreich Jan e01526208\\ Holzberger Fabian e11921655 }
\title{Machine Learing Exercise 0}
\maketitle

\medskip

\section{Introduction}
Two datasets are analyzed, one for Classification and a second one for regression. The Datasets were choosen such that they have different characteristics. The Characteristics and more information about the datasets is listed in the table below.
\newline

\begin{tabular}{lll}
\hline
Characteristic 		& Hepatitis 	& Airline   \\
Data Type 		& Multivariate 	& Multivariate \\
Attribute Type  	& Integer,Real  & Integer, String\\
Associated Tasks	& Classification & Regression \\
Number of instances     & 155 		 & 539383\\
Number of Attributes    & 19 		& 8\\
Missing Values 		& Yes 		& No \\
\hline
\end{tabular}
\newline



\section{Hepatitis Dataset}

\section{Airline Dataset}

\end{document}
\grid
\grid
